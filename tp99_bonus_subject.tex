\documentclass[10pt]{article}
\usepackage{macros}
\usepackage[a4paper,margin=2.5cm]{geometry}

%----------------------------------------------------
% Paramétrage de la fiche
%----------------------------------------------------
\sequence{Programmation en langage C++}
\seqlogo{\faCode}
\titrefiche{TP - BONUS}
\version{v1.0}
\dateversion{28.09.22}
\type{TP}

%----------------------------------------------------
% Définition des pieds et têtes de page
%----------------------------------------------------
% Pour toutes les pages
\pagestyle{fancy}
\fancyhead[L]{\seqlogo\ \sequenceVal}
\fancyhead[R]{\titreficheVal}
\fancyfoot[L]{BTS SNIR TS1 - Lycée Louis Rascol, Albi}
\fancyfoot[R]{\ccbyncsaeu}
\fancyfoot[C]{\thepage\ / \pageref{LastPage}}
% Pour la première page
\fancypagestyle{firstpage}{%
  \lhead{}
  \rhead{}
  \renewcommand{\headrulewidth}{0pt}
}

%----------------------------------------------------
% Début du document
%----------------------------------------------------
\begin{document}
\maketitle
\thispagestyle{firstpage}

%-----------------------
% Exo 1
%-----------------------
\section{Racines Carrées}
Écrire un programme qui calcule les racines carrées de nombres fournis en donnée.\\
Il s’arrêtera lorsqu’on lui fournira la valeur 0. Il refusera les valeurs négatives. Son
exécution se présentera ainsi :

\begin{textcode}
    donnez un nombre positif : 2
    sa racine carrée est : 1.414214e+00 donnez un nombre positif : -1
    svp positif
    donnez un nombre positif : 5
    sa racine carrée est : 2.236068e+00 donnez un nombre positif : 0
\end{textcode}

\begin{noteblock}
    Rappelons que la fonction sqrt fournit la racine carrée (double) de la valeur (double) qu’on lui donne en argument.
\end{noteblock}


%-----------------------
% Exo 2
%-----------------------
\section{Jeu de cartes}
On dispose d'un jeu de 52 cartes numérotées de 0 à 51.

\smallskip
Proposer deux instructions de type « alternative » qui, compte-tenu du numéro de la carte, précisent respectivement la couleur (pique, cœur, carreau, trèfle dans cet ordre) et la valeur de la carte (1, 2, ..., 10, valet, dame, roi).

Exemple de sortie console :

\begin{textcode}
    Saisissez un numero de carte : 5
    6 de pique

    Saisissez un numero de carte : 23
    valet de coeur

    Saisissez un numero de carte : 44
    6 de trèfle
\end{textcode}

\end{document}